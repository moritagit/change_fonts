\documentclass[dvipdfmx, a4paper, 14Q, fleqn]{jsarticle}

\input{/Users/User/Documents/TeX/preamble/preamble}

\usepackage[deluxe]{otf}
\usepackage[noalphabet]{pxchfon}

\setminchofont{ipag.ttf}											%明朝M = IPAゴシック
%\setlightminchofont{}
%\setmidiumminchofont{}
\setboldminchofont{GenEiGothicM-SemiBold.ttf}			%太明朝 = 源瑛ゴシックM中太
\setmediumgothicfont{GenEiGothicN-U-KL.otf}				%中ゴシック = 源瑛ゴシック特太
\setboldgothicfont{Kazesawa-Extrabold.ttf}					%太ゴシック = Kazesawa特太
\setxboldgothicfont{GenShinGothic-Heavy.ttf}				%極太ゴシック = 源真角ゴシック
\setmarugothicfont{HGRSGU.ttc}								%丸ゴシック = 創英角ゴシック

\begin{document}
\part*{フォントいじり祭}
各ウェイトに色んな極太フォントを設定して比較していく


IPAゴシック

{\mcfamily\bfseries 源瑛ゴシックM中太}

{\gtfamily\mdseries
源瑛ゴシック特太
}

{\gtfamily\bfseries
Kazesawa特太
}

{\gtfamily\ebseries
源真角ゴシック極太
}

{\mgfamily
創英角ゴシック
}


\part*{\sffamily\gtfamily\mdseries\huge \textbf{\textsf{2017}}年度 東京大学 前期 物理}		%源瑛ゴシック特太

{\mcfamily\bfseries
本文です~~~。
ここは源瑛の中太。
わーー
なに書いたらいいかわからないけど
本文との比較用平文。
解説からコピペしてきたいね

一応段落替え
}


\part*{\sffamily\gtfamily\bfseries\huge 2017年度 東京大学 前期 物理}							%Kazesawa特太

\part*{\sffamily\gtfamily\ebseries\huge \textbf{\textsf{2017}}年度 東京大学 前期 物理}		%源真角ゴシック


本文です~~~。
ここはIPAです。
わーー
なに書いたらいいかわからないけど
本文との比較用平文。
解説からコピペしてきたいね

一応段落替え

\part*{\sffamily\mgfamily\huge \textbf{\textsf{2017}}年度 東京大学 前期 物理}					%創英角ゴシック

以上を見た感じ、

\begin{itemize}
\item 本文 $\cdots$ 源瑛(ほんとは日へん)中太Mゴシック
\item タイトル類 $\cdots$ 源瑛(ほんとは日へん)特太ゴシック
\end{itemize}

がよいかと思われます。


\end{document}